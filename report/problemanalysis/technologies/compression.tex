An obvious issue with digital steganography is the bandwidth available in the information carriers, especially static data formats like images. For example, let's say we have a 100x100 image with 3 color channels (aRGB). If we use the least significant bit method and use the two least significant bits, the amount of bits available in this image is:
\begin{equation}100*100*8*2=160000\end{equation}
This effectively means that we can fit about 20 kilobytes of data in this image. Of course this is not a problem if we are simply hiding text messages, since this image would bit about 20.000 UTF8 characters, but one might want to hide files within the image. Whether it's text or files being hidden in the file, compression is a very useful technology, as it allows us to fit more data inside the image.

We can split this technology into two groups: lossy and lossless compression. In lossy compression, an algorithm finds redundant information and removes it. This is the same method used in the JPEG image format, where images are of noticeably worse quality after compression. Lossless compression also finds redundant information, but does not eliminate it. It utilizes statistical redudancy by checking for repeating patterns in the information. For example, images often have repeating pixels. If we have 100 green pixels written as "green pixel, green pixel, green pixel, ..." in the image, the compression algorithm can rewrite it to "100xgreen pixel". This is what the DEFLATE algorithm does, which is implemented in PNG among other file formats. 

In conclusion, there is no reason to not use data compression in a steganography tool, as long as a lossless algorithm like DEFLATE is utilized. 