Video files as a carrier media offers can be seen as a combination of multiple pictures in sequence, often with one or more accompanying audio tracks. As such it offers many of the same possibilities for hiding data as has been previously discussed in \ref{Pictures} and \ref{Audio} but also offers some new possibilities native to the video format. As an example of the former, it is possible to simply treat each frame of the video as a separate picture to perform steganography on, though even this offers certain new possibilities since the presence of possibly thousands of frames means that you might be able to spread any changes to the pixels over multiple frames making the changes to any single frame insignificant. The multitude of frames in a video can also serve to obscure changes to a single frame though this requires some care since it is necessary to consider not only how any changes fit into the actual frame but also how it fits with the surrounding frames. Especially if change is made to a otherwise static homogeneous area, the sudden change can become more obvious, and as such it is recommended to keep any changes to areas with change or motion between frames.
Other techniques could make use of the attributes of a the video codec such as motion vectors, or use the correlation between image and audio, but limited research has been done in these areas.
