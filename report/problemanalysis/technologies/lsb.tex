A commonly used method in digital steganography is the Least Significant Bit (LSB) method. The basic principle of this method is to make small changes to a large amount of data. LSB is useful in data formats where small change don't cause big differences. Embedding hidden messages in text would for example be a bad idea, as changes would be apparent, but on an image such changes may not be apparent. In a lossless image format like PNG (see \autoref{PNG}), the decompressed data is represented as a 2D matrix of pixels. Depending on the pixel format (sRGB, aRGB or perhaps even grayscale) each pixel has a set amount of color channels, where each value is a byte (8 bits), which determines the intensity of the color channel on that specific pixel. Of course the bit depth is not necessarily always 8 bits, but this is the case in most images shared over the internet. The purpose of the LSB method is to change the least significant bit in each of these bytes, and perhaps even the second least significant bit as well. This means that the decimal value may only change by 1 if modifying one bit and by 3 if modifying two least significant bits.

For example, say we want to embed the character '\textbf{*}' in a message. This ASCII character has a decimal value of 42, which can be represented as an 8-bit binary number:
\begin{equation}
00101010\label{beginmsg}
\end{equation}
Since this message is 8 bits long, we would need 8 bytes when modifying one bit and 4 bytes when modifying two bits. In the former case, this means that we would need two aRGB pixels, as this gives us 2 * 4 bytes to hide our message in. For example, let's say we want to hide our image in the two pixels:
\begin{equation}
\begin{split}
11111111\quad01001111\quad10110000\quad10000111\\
11111111\quad10010110\quad10001110\quad10000001 
\end{split}
\end{equation}
With the LSB method, we now need to look at the least significant bit in each of these bytes, and negate the bit if it does not correspond to the bit we want to insert. When the message is embedded in the original pixels, the produced result is:
\begin{equation}
\begin{split}
1111111\textbf{0}\quad0100111\textbf{0}\quad1011000\textbf{1}\quad1000011\textbf{0}\\
1111111\textbf{1}\quad1001011\textbf{0}\quad1000111\textbf{1}\quad1000000\textbf{0} 
\end{split}
\end{equation}
The embedded message can then be extracted by simply combining each least significant bit of the 8 bytes. After doing this, we end up with the message seen in \autoref{beginmsg}.

When comparing the pixel values of a modified image (now referred to as a stego-image) and the original image, the change is clear on the bit/byte level, but not be apparent to a human comparing the two images. This means that LSB is prone to steganalysis (see \autoref{Steganalysis}). With the stego-image and original image in possession, it is trivial for a piece of software to detect that one of the images have been edited. In addition, the LSB method will typically increase the size of the image, since adding noise to the least significant bits increases the amount of colors and can affect lossless compression used in formats such as PNG.

\subsection{Graph theoretic approach}
While the method described in the previous section is a sufficient steganography method, one might want a method that is less prone to steganalysis. After all, the point of steganography is to hide the fact that communication is taking place. In the method previously described, pixels are overwritten and the color frequencies are changed, but with a graph-theoretic approach it is possible to exchange pixels, so that the color frequencies are similar when comparing the original and stego-image. Depending on the size of the input message and the picture, it might be possible to hide a message in a picture without overwriting any pixels at all.