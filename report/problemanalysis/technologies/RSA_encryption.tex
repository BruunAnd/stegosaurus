The Rivest Shamir Adleman encryption, hencefourth reffered to as the RSA encryption is one of the first public key cryptosystems, and is used all over the world for transmitting data in a safe and secure way.
First published by Ron Rivest, Adi Shamir and Leonard Adleman in 1977, the RSA encryption uses a public key system which allows for the person in question to encrypt data, using a publicly declared encryption key, which works independently from the decryption key. The data can then be decrypted by anyone who has the corresponding private key, which; if it wasn't obvious, should be kept as private as possible.

RSA encryption also makes use of prime factorization, and the inherently difficult nature of figuring out which two numbers multiplied has been used in order to encrypt the message or data in question.

In steganography, one strives to conceal data within other data, where as the goal of cryptography is to hide data in plain sight by making it unreadable.
How ever one does not necessarily entail the exclusion of the other.

Especially when using steganography in order to transmit sensitive information the user might be interested in encrypting the data stored using steganography in the event that the data falls into the wrong hands.
This could be achieved using two different RSA encryptions with two agreed upon public encryption keys.

http://mathworld.wolfram.com/RSAEncryption.html