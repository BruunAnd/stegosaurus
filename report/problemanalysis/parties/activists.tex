When citizens of a country become victims of human rights violations, most citizens would turn to their governmentally run police force or a qualified equivalent hereof. However, when that same governmental system is what is responsible for human rights being violated in the first place, then a citizen has no choice but to seek help by attempting to get a message out beyond the boundaries of said country.

In the article “Landet, hvor det er forbudt at gå med røde bukser”\cite{rodebukser}, several North Korean citizens talk about their government not allowing their citizens the simple pleasures of life, such as dressing the way they want.

Though the North Korean government is losing it’s grip, on the North Korean citizens, this recent change in the power distribution, would not have been possible without the foreign influence that has gradually sifted through to North Korea, which can in part be owed to people daring enough to smuggle goods and knowledge into their country from beyond the borders of North Korea despite the wishes of the North Korean government.

It takes people being willing to either talk, against the will of their government or being willing to become refugees thusly leaving behind everyone and everything the person in question had, just so that other people beyond the boundaries of for example North Korea might learn the truth of North Korean citizenship.

A possible solution could be to offer North Korean citizens an alternative. Through steganography, North Korean citizens might be able to ‘below the radar’ send messages to fellow citizens and other countries alike through alternative carrier medias such as music files, pictures and so on.