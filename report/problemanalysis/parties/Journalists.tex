A journalist wants to spread a story to multiple people and wants the sources to be credible. They therefore often have to search for people involved in the subject they are writing about, to have a primary source, which is a person that is directly evolved in the subject or for example written a document about it. The contact or source, can live in a dangerous environment, where secrecy is a vital part of communication. Often the best option is to keep offline, away from the internet and meet the source in person, but this is not always an option or very impractical.\cite{secret} This is where steganography or similar methods come in handy. As described in section ?? , Victims of human rights violation, there are governments and groups that want to retain information from public knowledge and disclosing the information could lead to prison or death. A case, Goodwin v. United Kingdom, in 1996, the European Court of Human Rights stated that, 

"Protection of journalistic sources is one of the basic conditions for press freedom ... Without such protection, sources may be deterred from assisting the press in informing the public on matters of public interest. As a result the vital public-watchdog role of the press may be undermined and the ability of the press to provide accurate and reliable information may be adversely affected"\cite{case}

meaning that the public should not be afraid of sharing their information to the media. Because of the danger of disclosing information in some countries, journalists can have a hard time finding sources, unless they can communicate in a safe circumstances. The problem is that most of the sources have a minimal amount technological knowledge. As said in "Investigating the Computer Security Practices and Needs of Journalists", the sources rarely have access to security technology and/or do not understand how to use security software. The journalist, according to this report, often use what the sources feel comfortable with. This means that the jounalist may use some technology that are unsafe, knowingly.\cite{journalists}

This leads to steganography, encryption and similar forms of concealment, that can ensure safety or drastically reduce the risk of getting uncovered. If the people feel safe, they are more likely to share their stories and the journalists have the possibility to pass on the information to the public. Programs that can keep the communication secret, would be a great asset to the journalist, but the program have to be easily obtainable and easy to use.