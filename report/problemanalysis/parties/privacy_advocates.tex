Some steganography users are difficult to place in a specific group. Privacy advocates, for lack of a better name, are those who want to protect themselves from prying eyes, whether those are government bodies, criminals with ill intentions or simply other common people. As such, a subset of these advocates may be referred to as IT-activists, as their sole intention with steganography and cryptography is to counter mass surveillance. With that said, their interest in steganography is obviously broad. Some of these people may have an interest in hiding certain files on their file system, while others may simply use it to send messages to their friends. The motivation for these groups may seem vague or difficult to understand for somebody who does not care about privacy, but the matter is obviously quite important for the privacy advocates. 

An argument often used against this group is the ‘nothing to hide’ argument. People who support this argument will argue that mass surveillance does not threaten the privacy of individuals, unless government agencies uncover illegal activities by this individual, in which case they don’t have the right for privacy. On the other hand, privacy advocates will argue that the argument implies full trust in the state, which they argue can be hard to assume, even in fully developed nations. Nonetheless, some people want to hide from criminals trying to misuse their personal information, and tools like steganography and cryptography can be useful in this scenario.