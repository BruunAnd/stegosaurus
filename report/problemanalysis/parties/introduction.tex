To enable a proper definition of a problem to be solved, it is necessary to examine the stakeholders in question, in regards to steganography. As such the following analysis will cover two major areas of interest. The interest in using steganography, and the interest in discovering steganography using steganalysis. These can be seen as opposing, but not mutually exclusive interests.
 
Steganography users have an interest in hiding information, this information can be applied in many areas, and has a usefulness for not only personal or criminal use, but also for government or corporate offices, where secret information can be sent with reduced risk of unauthorized interception. This information could range from classified information such as state secrets to simply private information such as citizens taxes.

Steganalysis users have an interest in discovering hidden data, this is mostly done with statistical analysis of the data structure in question, and is mainly used by governments or corporate offices. This may heighten a nation's or corporate office's security, and make them able to preemptively make decisions to help their own interests such as increasing security based on intercepted messages and thus prevent an attack.

The stakeholder analysis will be a basis for further research, as different parties have different requirements to a potential steganography program.