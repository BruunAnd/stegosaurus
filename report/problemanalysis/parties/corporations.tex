Steganography is also used by many corporations. A common use is in watermarking their products. In this way they hide information in the product which can be used to identify the specific item and possibly the rightful possessor. This can help prevent unauthorized sharing of videos by identifying who is responsible for the leak and thus punish the responsible party and possibly blacklisting said person to prevent future leaks.

However, other uses of steganography pose significant risks for corporate espionage as it enables the exchange of covert messages in otherwise harmless media. Many companies monitor employee emails, and would thus be able to identify obvious encrypted data but not data obscured in harmless media such as family photos.