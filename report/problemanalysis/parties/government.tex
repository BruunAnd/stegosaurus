The government is divided into a variety of agencies that, in the course of fulfilling their obligations, use steganography and steganalysis. These government departments have different needs, because of the different nature of their overall assignments, and is divided into their own segment of steganography and steganalysis.

%National security agencies
Agencies like NSA, and the danish PET, can hypothetically use steganography to hide messages from agents operating undercover in foreign countries. This will enable the agents to hide in plain sight, as they spy for the interests of the particular nation. That nations spy on each other is seen in many examples. fx. the infiltration of the Syrien main internet router in 2012. However not much is known on how governments apply steganography and steganalysis, as this information is closely protected in what each nation defines as a matter of “national security”, and the disclosed information here is only known because of the actions of Edward Snowden, who conducted the biggest national security breach in American history.\cite{governmentsurveillance} However, this information is mostly about information gathering, and country on country espionage, and doesn’t cover the area of steganography.

It is however proven by FireEye, that an alleged Russian hacker group is responsible for a malware named “Hammertoss”.\cite{spionertwitter} This malware, is directly connected to twitter, where the group can send Steganographic images to a specific account, and by doing that, activate the malware on the infected computers. The malware will then send information, documents and files to a cloud server, where the perpetrators can download the data. This method could theoretically be used in a government perspective, where data can be gathered from personal computers or corporation server systems, if this data is not encrypted.
At the same time, this example, can also determine that national security agencies have to increase their efforts to detect steganography on pages like twitter, and ultimately, this means that the national security agencies have to take part in both steganography and steganalysis.

Law enforcement forensic methods are equally hard to come by. This may be because of the nature of IT, and the abilities of IT-criminals to turn that information to their own advantage. It is however, safe to assume that government authorities like the police, are able to use steganalysis to uncover criminals, as these information can be crucial for a court proceeding.
As such, the government has an interest in steganalysis, and therefore in steganography, because of a need to uphold the law of the nation in question.