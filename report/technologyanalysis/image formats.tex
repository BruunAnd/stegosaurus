Image file types:

There are many different image file types to choose from these days, some easier to implement stenography on than others. In this section we will describe a few image file types along with a general explanation of why images are subject to stenography.

JPEG File Interchange Format (JFIF) is an image format that is a common standard in most modern digital capture devices such as digital cameras and image software alike. JFIF uses the Joint Photographic Experts Group (JPEG) compression method (for more info about compression see chapter \ref{compression}).

Portable Network Group (PNG) is another image format widely used eight-bit paletted images because of the support for transparency for every palette color and 24-48 bit truecolor with and without alpha channels.

Windows bitmap (BMP) is a format that is more focused around files within the Microsoft Windows OS. While BMP files are usually larger in file size due to commonly being uncompressed, their advantage lies in simplicity and ease of use.

Graphical Interchange Format (GIF) is widely used to make small animations often to express humor, for example through the use of "memes". GIFs have limited support for colors and transparency compared to other mentioned image file formats. For example GIFs can only use an 8-bit palette or 256 colors.

All these image file types share some common principles, for example the use of pixels. Pixels are essential when talking about images and these pixels are also what is used to hide messages in the images. The difficult thing when hiding messages comes when altering the pixels because the pixels are essentially the visual representation of the image and therefore a pixel cannot be altered too much in order to avoid detection.

Pixels are the smallest element in any digital image. They can be subdivided to make larger resolutions which means having more pixels within the same area. This would mean that each pixels would be divided into smaller pixels with the same value.\cite{ThePixel}